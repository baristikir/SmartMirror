\section{Grundidee}
\paragraph{Wie wir auf die Idee gekommen sind...}
Auf die eigentliche Idee ist Barsi Tikir eines morgens gekommen als er im Bad stand. Während er sich für den Tag fertig machte und auf seinem Handy noch seine passsende Bahnverbindung raus suchte, dachte er darüber nach, wie praktisch es wäre, wenn man seine Bahnverbindungen nicht mühsam im Handy über die BVG App suchte müsste, sondern diese im irgendwie direkt präsent wäre. Als er darufhin in den Spiegel blickte kam im die Idee.. ein Spiegel  der seine Bahnverbindung anzeigen könnte. Morgens In den Spigel gucken, beim Zähneputzen, Haare machen, SChminken, usw.... Warum nicht die Zeit auch geich nutzen für ein kleines Update, was in der Welt gerade so passiert oder wann der nächste Bus zur Arbeit fährt. Nach kurzer Recherche fand er auch eine passende Bauanleitung für einen solchen Spiegel. Jedoch gab es neben dem Zusammenbau noch das Problem etwas  sinnvolles auf dem Display anzuzeigen. kleinere Projekt mit Uhrzeit und Wetter gab es bereits, als Beispiele zum nach-programmieren. 
\\\
Ein paar Wochen später stieß er auf den Paulaward. Als wir uns trafen, erzählte Baris über die Idee und den Paul Award. Zusammen haben wir dann ein Konzept entwickelt, welches möglichst viele Infomationen aus verschiedenen Bereichen anzeigen kann. Da es jedoch schon sehr viele Dienste, wie zum Beispiel Google Kalender, ToDoIst, BVG App und so weiter gibt, wollte wir nicht einen weiteren doppelten Dienst entwickeln, sonder auf diese Zugreifen, sodass wir die Daten, welche wir hinter diesen Diensten bereits abgelegt hatten, nutzen können. So entstand die Gesamtidee vom SmartMirror.

\subsection{Spiegel mit integriertem Display}
Ein Spiegel mit integriertem Display ist wohl nichts neues. Die Idee ist einfach hinter einem Einwegspiegelglass ein Display zu montieren, sodass dieses durch das Glass durch scheint und man dies auf der anderen Seite des Spiegels sehen kann. Von der Anderen Seite wirkt das Galss spiegelnd, wodurch es ganz Normal als Spiegel genutzt werden kann.
\subsection{Konzept Benutzerfreundlichkeit}
Die eigentliche Idee dabei ist den Spiegel nicht einfach nur spiegeln zu lassen oder die Uhrzeit anzeigen zu lassen, sondern ihn "smart" zu machen, sodass man ihn praktischen und effizient Nutzen kann. Man geht morgens Zudem war uns auch wichtig, dass es \textbf{Benutzerfreundlich} ist, da nicht jeder das Know-How hat sich einen Spiegel für seine Eigenen Bedrüfnisse zusammen zu bauen bzw. zu programmieren.