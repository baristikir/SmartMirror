\documentclass[a4paper,11pt]{article}

% packages
% Sprache Deutsch
\usepackage[english,german]{babel}
% 
\author{Baris,Tikir, Leon Dodrimong}
\title{Smart Mirror}

\renewcommand*\contentsname{Inhaltsverzeichnis}
\begin{document}
\maketitle
\newpage
\tableofcontents
\newpage


\section{Grundidee}

\subsection{Konzept Benutzerfreundlichkeit}
\subsection{Spiegel mit integriertem Display}


\section{Aufbau}
\subsection{Strukturierung}
Notizen: Dagramme und erklärungen der Funktionsweise\\\
\subsection{Materialien}
Der Spiegel besteht aus einem Einwegspiegelglass, welches von einer Seite spiegelt und von der anderen Seite reflektiert. Außerdem haben wir einen Holzrahmen, welcher aus Fassung für das Spiegelglass dient, genutzt. Für die Technik haben wir ein Display für die Anzeigen , ein RasperryPi für die Steuerung und die nötigen Verbindungskabel, wie Spannungsversorgung und Videokabel (HDMI) zur Übertragung der Videosignals zum Display, eingesetzt.\\\
Für die optinalen Erweiterungen würde man je nachdem welches Feature gewünscht ist, noch eine Picamera (für Facerecognition), RasperryPi Bewegungssensor (Bewegungserkennung)\footnote{\textit{ Raspberry Pi Infrarot Bewegungsmelder:} https://www.reichelt.de/raspberry-pi-infrarot-bewegungsmelder-hc-sr501-rpi-hc-sr501-p224216.html?\&nbc=1}, Gesture Sensor (Gestiksteuerung)\footnote{\textit{3D Gesture Tracking Shield for Raspberry Pi:}  http://wiki.seeedstudio.com/3D-Gesture-Tracking-Shield-for-Raspberry-Pi-MGC3130/} oder ein kleines Mikrophone zur Sprachsteuerung \footnote{\textit{Ansteckmikrofon über Klinke:} https://www.amazon.de/dp/B073GJQKL1/ref=psdc\_1384055031\_t1\_B07WQFNVVQ}
\\\\\\
Notizen: Später eintragen welches Raspi Modell, Display, Glass verwendet wurde

\section{Erweiterungen}
\subsection{API's}

\begin{itemize}
\item Google Kalender
\item Todoist
\item BVG
\item Wetter
\end{itemize}
\subsection{Hardware}
\begin{itemize}
\item Face Recognition
\item Voice Control
\item Bewegungssensor
\end{itemize}


\section{Über uns}
Wir sind zwei Studenten, Baris Tikir und Leon Dodrimong, von der HTW - Hochschule für
Technik und Wirtschaft Berlin aus dem Fachbereich der Ingenieurwissenschaften und
Technik. 

\section{Anhang}
\subsection{Verweise}


\end{document}